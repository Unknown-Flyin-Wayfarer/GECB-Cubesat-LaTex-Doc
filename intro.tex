\chapter*{Introduction}

A CubeSat is a miniaturized version of a satellite that is composed of cube units (U) of 10 cm by 10 cm by
10 cm. Government Engineering College Barton Hill is developing this 1U CubeSat. Its mission is to serve as
a platform for university students to learn to design and develop a 1U CubeSat bus capable of being reused in
subsequent missions. A key system in this bus is the Attitude Determination and Control System (ADCS). The
ADCS is responsible for determining and controlling a satellite’s orientation in orbit. One of its first requirements is
to reduce the rotation imparted by the CubeSat’s deployer into a more stable motion, a pointing operation often
referred to as detumble. This detumbling process can be managed either passively or actively. Passive control
mechanisms are simple, often requiring no moving parts or power. For example, a passive control system could use
permanent magnets or hysteresis rods to magnetically control orientation. This has several major drawbacks such as
a limited attitude pointing accuracy of about +/- 10 degrees about Earth’s magnetic field, constraining all other
pointing requirements for power, communications and other sensors [1] . Another approach uses active control
mechanisms like reaction wheels or magnetorquers to control the spacecraft’s attitude and orientation. While
somewhat more complicated, these systems allow the satellite to be more precisely controlled. This cubesat's planned
ADCS will utilize an active control system using magnetorquers, also called torque rods, to manage detumble and
some pointing requirements. This paper will lay out the design, assembly and testing of these torque rods.